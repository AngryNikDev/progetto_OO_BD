\section{Progettazione fisica}
\subsection{Creazione del database}
\begin{lstlisting}
CREATE DATABASE company WITH OWNER postgres;
\end{lstlisting}

\subsection{Creazione dello schema}
\begin{lstlisting}
CREATE SCHEMA projects_schema;
\end{lstlisting}

\subsection{Attivazione estensioni}
Le estensioni possono essere attivate solo da un superuser (\textit{postgres} in questo caso); \textit{project\_admin} non può essere superuser altrimenti avrebbe gli stessi privilegi di \textit{postgres}.

Le estensioni vengono create sullo schema \textit{projects\_schema} così che l'utente \textit{project\_admin} ne abbia l'accesso.\meskip
Estensione utilizzata per la cifratura delle password, usata per l'autenticazione lato client.
\lstinputlisting{../basi_di_dati/extension/enable_crypt.sql}\medskip
Estensione utilizzata per generare codici univoci causali.
\lstinputlisting{../basi_di_dati/extension/enable_uuid.sql}

\subsection{Gestione ruoli e permessi}
All'utente \textit{project\_admin} è affidata la gestione dello schema \textit{projects\_schema}. Si occuperà di tutto, dalla creazione delle tabelle ai trigger \dots\\
Non essendo possessore dello schema, gli è impossibile cancellarlo.\sskip
Il numero massimo di connessioni è $1$.
\begin{lstlisting}
CREATE USER project_admin WITH 
    PASSWORD 'a'
    CREATEROLE
    CONNECTION LIMIT 1;

GRANT ALL PRIVILEGES ON SCHEMA projects_schema TO project_admin;
\end{lstlisting}\medskip
Utente utilizzato esclusivamente per effettuare l'accesso lato client.\\
Ha accesso in lettura solo in \textit{\baseemp} e \textit{\projectsalaried} perché ha bisogno di \textit{email} e \textit{password}
\begin{lstlisting}
CREATE USER login_user WITH
	PASSWORD 'login';

GRANT SELECT ON base_emp, project_salaried TO login_user;
\end{lstlisting}
\newpage \noindent
Ruolo generico che accorpa permessi necessari a tutti gli utenti che ne "ereditano" il ruolo.
\begin{lstlisting}
CREATE ROLE base_emp_role;

GRANT USAGE ON SCHEMA projects_schema TO base_emp_role;

GRANT SELECT ON base_emp, career_log, works_at, laboratory, take_part, equipment, site 
	     TO base_emp_role;

GRANT SELECT(CUP, name, description, start_date, end_date, deadline) ON project 
	     TO base_emp_role;
\end{lstlisting}\medskip
\textit{junior\_user} e \textit{middle\_user} non hanno ulteriori permessi specifici.
\begin{lstlisting}
CREATE USER junior_user WITH 
	PASSWORD 'junior'
	IN ROLE base_emp_role;

CREATE USER middle_user WITH
	PASSWORD 'middle'
	IN ROLE base_emp_role;
\end{lstlisting}\medskip
L'utente \textit{senior\_user} accorpa i permessi di \textit{scientific\_referent} e \textit{scientific\_manager}.
\begin{lstlisting}
CREATE USER senior_user WITH
	PASSWORD 'senior'
	IN ROLE base_emp_role;

-- Possibilità di prendere parte ad un progetto e richiedere attrezzatura
GRANT INSERT ON take_part, equipment_request TO senior_user; 

-- Possibilità di abbandonare un progetto
GRANT UPDATE(end_date) ON take_part TO senior_user;

-- Modificare la richiesta di attrezzatura
GRANT UPDATE(specs, quantity) ON equipment_request TO senior_user;
\end{lstlisting}
\begin{lstlisting}
CREATE USER manager_user WITH
	PASSWORD 'manager'
	IN ROLE base_emp_role;	
\end{lstlisting}\medskip
Questi permessi sono condivisi sia dallo \textit{scientific\_referent} che dal \textit{manager}
\begin{lstlisting}
GRANT SELECT ON project,  
                purchase,  
                equipment_request,  
                project_salaried,  
                works_on, 
                remaining_project_funds 
             TO manager_user, senior_user;

GRANT INSERT ON equipment, 
                purchase, 
                project_salaried, 
                works_on 
             TO manager_user, senior_user;

GRANT DELETE ON equipment_request TO manager_user, senior_user;
\end{lstlisting}
Non essendo \textit{\baseemp}, non ne eredita il ruolo ma può visualizzare solo le informazioni dei progetti a cui lavora.
\begin{lstlisting}
CREATE USER project_salaried_user WITH
	PASSWORD 'proj_sal';

GRANT USAGE ON SCHEMA projects_schema TO project_salaried_user;

GRANT SELECT ON project_salaried, works_on, project
             TO project_salaried_user;
\end{lstlisting}

\subsection{Creazione domini}
\lstinputlisting{../basi_di_dati/domains/cf_type.sql}
\lstinputlisting{../basi_di_dati/domains/cup_type.sql}
\lstinputlisting{../basi_di_dati/domains/emp_type.sql}
\lstinputlisting{../basi_di_dati/domains/equipment_name_type.sql}
\lstinputlisting{../basi_di_dati/domains/name_type.sql}
\lstinputlisting{../basi_di_dati/domains/password_type.sql}
\lstinputlisting{../basi_di_dati/domains/salary_type.sql}

\newpage
\subsection{Creazione tabelle}
\noindent \textbf{Tabella \baseemp}
\lstinputlisting[lastline=14]{../basi_di_dati/create_tables/base_emp.sql}\bigskip

\noindent \textbf{Tabella \careerlog}
\lstinputlisting[lastline=33]{../basi_di_dati/create_tables/career_log.sql}\bigskip

\newpage
\noindent \textbf{Tabella laboratory}
\lstinputlisting{../basi_di_dati/create_tables/laboratory.sql}\bigskip

\noindent \textbf{Tabella \takepart}
\lstinputlisting{../basi_di_dati/create_tables/take_part.sql}\bigskip

\noindent \textbf{Tabella \projectsalaried}
\lstinputlisting{../basi_di_dati/create_tables/project_salaried.sql}\bigskip

\newpage
\noindent \textbf{Tabella \workson}
\lstinputlisting{../basi_di_dati/create_tables/works_on.sql}\bigskip

\noindent \textbf{Tabella site}
\lstinputlisting{../basi_di_dati/create_tables/site.sql}\bigskip

\noindent \textbf{Tabella equipment}
\lstinputlisting{../basi_di_dati/create_tables/equipment.sql}\bigskip

\newpage
\noindent \textbf{Tabella purchase}
\lstinputlisting{../basi_di_dati/create_tables/purchase.sql}\bigskip

\noindent \textbf{Tabella \worksat}
\lstinputlisting{../basi_di_dati/create_tables/works_at.sql}\bigskip

\newpage
\noindent \textbf{Tabella \equipmentrequest}
\lstinputlisting{../basi_di_dati/create_tables/equipment_request.sql}\bigskip

\noindent \textbf{Tabella project}
\lstinputlisting{../basi_di_dati/create_tables/project.sql}\bigskip

\newpage
\subsection{Trigger e trigger function}\label{trigger}
La descrizione dei seguenti trigger è riportata nella \hyperref[vincoli_inter_relazionali]{\textcolor{red}{\underline{sezione dei vincoli inter-relazionali}}}\meskip
\noindent \textbf{check\_equipment\_funds}
\lstinputlisting[firstline=3]{../basi_di_dati/triggers/check_equipment_funds.sql}\bigskip

\noindent \textbf{check\_funds\_to\_hire}
\lstinputlisting[firstline=3]{../basi_di_dati/triggers/check_funds_to_hire.sql}\bigskip

\newpage
\noindent \textbf{check\_laboratory\_assignment}
\lstinputlisting[firstline=4]{../basi_di_dati/triggers/check_laboratory_assignment.sql}\bigskip
\newpage
\noindent \textbf{check\_project\_assignment}
\lstinputlisting[firstline=4]{../basi_di_dati/triggers/check_project_assignment.sql}\bigskip

\newpage
\noindent \textbf{check\_works\_on\_date}
\lstinputlisting[firstline=2]{../basi_di_dati/triggers/check_works_on_date.sql}\bigskip

\noindent \textbf{crypt\_password}
\lstinputlisting{../basi_di_dati/triggers/crypt_password.sql}\bigskip

\newpage
\noindent \textbf{employee\_hiring}
\lstinputlisting[firstline=3]{../basi_di_dati/triggers/employee_hiring.sql}\bigskip

\noindent \textbf{fire\_emp}
\lstinputlisting{../basi_di_dati/triggers/fire_emp.sql}\bigskip

\noindent \textbf{max\_lab\_in\_project}
\lstinputlisting[firstline=2]{../basi_di_dati/triggers/max_lab_in_project.sql}\bigskip

\noindent \textbf{update\_emp\_type}
\lstinputlisting[firstline=4]{../basi_di_dati/triggers/update_emp_type.sql}\bigskip

\newpage
\noindent \textbf{take\_part\_no\_duplicates}
\lstinputlisting[firstline=2]{../basi_di_dati/triggers/take_part_no_duplicates.sql}\bigskip

\noindent \textbf{validate\_equipment\_request}
\lstinputlisting[firstline=2]{../basi_di_dati/triggers/validate_equipment_request.sql}\bigskip

\noindent \textbf{works\_at\_no\_duplicates}
\lstinputlisting[firstline=2]{../basi_di_dati/triggers/works_at_no_duplicates.sql}\bigskip

\noindent \textbf{end\_project}
\lstinputlisting[]{../basi_di_dati/triggers/end_project.sql}

\newpage
\subsection{Procedure e funzioni}
\noindent \textbf{base\_emp\_login}\sskip
Effettua il login di un \textit{\baseemp} con \textit{email} e \textit{password}.

Restituisce il tipo (junior, middle, senior, manager) se il login è avvenuto con successo, \textit{NULL} altrimenti.
\lstinputlisting[firstline=3]{../basi_di_dati/functions/base_emp_login.sql}\bigskip

\noindent \textbf{project\_salaried\_login}\sskip
Effettua il login di un \textit{\projectsalaried} con \textit{email} e \textit{password}.

Restituisce vero o falso a seconda della riuscita del login.
\lstinputlisting[firstline=2]{../basi_di_dati/functions/project_salaried_login.sql}\bigskip

\newpage
\noindent \textbf{buy\_equipment}\sskip
Prende in ingresso un codice di una richiesta attrezzatura e il prezzo per singolo prodotto.

Soddisfa la richiesta effettuando gli acquisti richiesti.
\lstinputlisting{../basi_di_dati/functions/buy_equipment.sql}\bigskip

\newpage
\noindent \textbf{change\_lab\_afference}\sskip
Prende in ingresso il \textit{cf} di un impiegato e il codice di un laboratorio nel quale spostare l'impiegato.

Se l'impiegato sta lavorando in un laboratorio diverso rispetto al quale lo si vuole spostare, lo rimuove prima da quello vecchio (aggiornando la data di fine \textit{end\_date} a \textit{CURRENT\_DATE}).

Successivamente si inserisce in \textit{\worksat} l'impiegato per quel laboratorio.
\lstinputlisting[firstline=2]{../basi_di_dati/functions/change_lab_afference.sql}\bigskip

\newpage
\noindent \textbf{check\_seniority}\sskip
Controlla il grado di anzianità degli impiegati e li aggiorna nel caso. Si suppone che il controllo dell'avanzamento di grado verrà fatto una volta al mese dopo il giorno di paga.
\lstinputlisting[firstline=2, lastline=62]{../basi_di_dati/functions/check_seniority.sql}\bigskip

\noindent \textbf{fire}\sskip
Licenzia l'impiegato passato come parametro, ossia inserisce una tupla in

\textit{\careerlog} con \textit{new\_role} = ' '
\lstinputlisting[firstline=2]{../basi_di_dati/functions/fire.sql}\bigskip

\noindent \textbf{hire\_project\_salaried}\sskip
Assume un impiegato a progetto. Se questo è già stato inserito nella tabella \textit{\projectsalaried}, verrà inserita solo la tupla in \textit{workson}.
\lstinputlisting{../basi_di_dati/functions/hire_project_salaried.sql}\bigskip

\newpage
\noindent \textbf{promotion\_to\_manager}\sskip
Promuove l'impiegato preso in input a \textit{manager}.

Se è gia \textit{manager} oppure è \textit{referente scientifico} o \textit{manager scientifico} non può essere promosso. \\
Altrimenti inserisce in \textit{\careerlog} con \textit{new\_role} = 'manager'
\lstinputlisting{../basi_di_dati/functions/promotion_to_manager.sql}\bigskip

\newpage
\noindent \textbf{revoke\_manager}\sskip
Declassa l'impiegato preso in input da \textit{manager} al ruolo che aveva precedentemente.

Se nel mentre che era \textit{manager} è avanzato di grado per anzianità, non dovrà essere declassato al ruolo precedente ma al nuovo grado (in base all'anzianità).
\lstinputlisting{../basi_di_dati/functions/revoke_manager.sql}

\subsection{Dizionario dei vincoli}
\subsubsection{Vincoli intra-relazionali}
\begin{tabular}{@{}| p{.3\textwidth} | p{.6\textwidth} |@{}} % base emp
	\hline
	\multicolumn{2}{|c|}{\textbf{\baseemp}}         \\
	\hline
	\textbf{Vincolo}   & \textbf{Descrizione}       \\
	\hline
	emp\_email\_unique & L'email è univoca          \\
	\hline

	\baseemp\_pk       & Vincolo di chiave primaria \\
	\hline
\end{tabular}\bskip
\begin{tabular}{@{}| p{.3\textwidth} | p{.6\textwidth} |@{}} % laboratory
	\hline
	\multicolumn{2}{|c|}{\textbf{laboratory}}                \\
	\hline
	\textbf{Vincolo}            & \textbf{Descrizione}       \\
	\hline
	lab\_pk                     & Vincolo di chiave primaria \\
	\hline
	site\_number\_fk            & Vincolo di chiave esterna  \\
	\hline
	cf\_scientific\_manager\_fk & Vincolo di chiave esterna  \\
	\hline
\end{tabular}\bskip
\begin{tabular}{@{}| p{.3\textwidth} | p{.6\textwidth} |@{}} % career log 
	\hline
	\multicolumn{2}{|c|}{\textbf {\careerlog}}                                                                                                     \\
	\hline
	\textbf{Vincolo}  & \textbf{Descrizione}                                                                                                       \\
	\hline
	emp\_career\_fk   & \begin{minipage}[t]{.6\textwidth}
		                    Vincolo di chiave esterna.\\
		                    Se viene eliminato l'impiegato, verrà eliminata anche la tupla in \textit{\careerlog}

	                    \end{minipage}                                       \\[26pt]
	\hline
	check\_new\_grade & \begin{minipage}[t]{.6\textwidth}
		                    Controlla che gli scatti di carriera siano coerenti.\\
		                    Ad esempio non è possibile passare da \textit{junior} a \textit{senior} senza essere diventati prima \textit{middle}.\sskip
		                    La stringa vuota come primo parametro significa che l'impiegato è stato assunto;\\
		                    come secondo invece, significa che è stato licenziato.\meskip
		                    \textbf{Casi possibili:}
		                    \begin{itemize}
			\item (' ', 'junior')
			\item ('junior', 'middle')
			\item ('middle', 'senior') \medskip
			\item ('junior', 'manager')
			\item ('middle', 'manager')
			\item ('senior', 'manager') \medskip
			\item ('manager', 'junior')
			\item ('manager', 'middle')
			\item ('manager', 'senior') \medskip
			\item ('junior', ' ')
			\item ('middle', ' ')
			\item ('senior', ' ')
			\item ('manager', ' ')

		\end{itemize}

	                    \end{minipage} \\[310pt]
	\hline
\end{tabular}\bskip
\begin{tabular}{@{}| p{.3\textwidth} | p{.6\textwidth} |@{}} % equipment request
	\hline
	\multicolumn{2}{|c|}{\textbf{\equipmentrequest}}                                                                                                             \\
	\hline
	\textbf{Vincolo}       & \textbf{Descrizione}                                                                                                                \\
	\hline
	equipment\_request\_pk & Vincolo di chiave primaria                                                                                                          \\
	\hline
	CUP\_fk                & \begin{minipage}[t]{.6\textwidth}
		                         \raggedright
		                         Vincolo di chiave esterna.\\
		                         Se viene eliminato il progetto (o modificato il cup), verrà eliminata (o aggiornata) anche la tupla in \textit{\equipmentrequest}
	                         \end{minipage}    \\[40pt]
	\hline
	lab\_code\_fk          & \begin{minipage}[t]{.6\textwidth}
		                         \raggedright
		                         Vincolo di chiave esterna.\\
		                         Se viene eliminato il laboratorio (o modificato il cup), verrà eliminata (o aggiornata) anche la tupla in \textit{\equipmentrequest}
	                         \end{minipage} \\[40pt]
	\hline
	check\_quantity        & Controlla che la quantità sia maggiore di $0$                                                                                       \\
	\hline
\end{tabular}\bskip
\begin{tabular}{@{}| p{.3\textwidth} | p{.6\textwidth} |@{}} % equipment
	\hline
	\multicolumn{2}{|c|}{\textbf{equipment}}                                                                          \\
	\hline
	\textbf{Vincolo} & \textbf{Descrizione}                                                                           \\
	\hline
	equipment\_pk    & Vincolo di chiave primaria                                                                     \\
	\hline
	lab\_code\_fk    & \begin{minipage}[t]{.6\textwidth}
		                   \raggedright
		                   Vincolo di chiave esterna.\\
		                   Se viene eliminato il laboratorio, la chiave esterna verrà posta a NULL.\\
		                   Se viene aggiornata la chiave di \textit{laboratory}, verrà aggiornata anche la chiave esterna.
	                   \end{minipage} \\[50pt]
	\hline
\end{tabular}\bskip
\begin{tabular}{@{}| p{.3\textwidth} | p{.6\textwidth} |@{}} % project salaried
	\hline
	\multicolumn{2}{|c|}{\textbf{\projectsalaried}}   \\
	\hline
	\textbf{Vincolo}     & \textbf{Descrizione}       \\
	\hline
	\projectsalaried\_pk & Vincolo di chiave primaria \\
	\hline
	emp\_email\_unique   & L'email è univoca          \\
	\hline
\end{tabular}\bskip
\begin{tabular}{@{}| p{.3\textwidth} | p{.6\textwidth} |@{}} % project
	\hline
	\multicolumn{2}{|c|}{\textbf{project}}                                                                         \\
	\hline
	\textbf{Vincolo}             & \textbf{Descrizione}                                                            \\
	\hline
	project\_pk                  & Vincolo di chiave primaria                                                      \\
	\hline
	check\_deadline              & Verifica che la deadline sia dopo la data di inizio (nel caso in cui ci fosse). \\
	\hline
	check\_end\_date             & Verifica che la data di fine progetto sia dopo la data di inizio.               \\
	\hline
	fk\_cf\_manager              & Vincolo di chiave esterna                                                       \\
	\hline
	fk\_cf\_scientific\_referent & Vincolo di chiave esterna                                                       \\
	\hline
\end{tabular}\bskip
\begin{tabular}{@{}| p{.3\textwidth} | p{.6\textwidth} |@{}} % purchase
	\hline
	\multicolumn{2}{|c|}{\textbf{purchase}}                                                                                                                                 \\
	\hline
	\textbf{Vincolo}    & \textbf{Descrizione}                                                                                                                              \\
	\hline
	cup\_type\_fk       & Vincolo di chiave esterna.

	Se viene eliminato il progetto, la chiave esterna viene posta a NULL, questo perché si vuole tenere traccia della data d'acquisto e del prezzo dell'attrezzatura.       \\
	\hline
	equipment\_code\_fk & Vincolo di chiave esterna.

	Se viene eliminato l'attrezzatura, verrà eliminata anche la tupla in purchase, dato che diventa inutile tenere traccia dell'acquisto di un prodotto che non si conosce. \\
	\hline
	check\_price        & Controlla che il prezzo sia maggiore di $0$                                                                                                       \\
	\hline
\end{tabular}\bskip
\begin{tabular}{@{}| p{.3\textwidth} | p{.6\textwidth} |@{}} % site
	\hline
	\multicolumn{2}{|c|}{\textbf{site}}           \\
	\hline
	\textbf{Vincolo} & \textbf{Descrizione}       \\
	\hline
	site\_pk         & Vincolo di chiave primaria \\
	\hline
\end{tabular}\bskip
\begin{tabular}{@{}| p{.3\textwidth} | p{.6\textwidth} |@{}} % take part
	\hline
	\multicolumn{2}{|c|}{\textbf{\takepart}}                                                                                                             \\
	\hline
	\textbf{Vincolo} & \textbf{Descrizione}                                                                                                              \\
	\hline
	date\_integrity  & Verifica che la data in cui il laboratorio termina di lavorare al progetto, sia dopo la data di inizio.                           \\
	\hline
	CUP\_fk          & \begin{minipage}[t]{.6\textwidth}
		                   \raggedright
		                   Vincolo di chiave esterna.\\
		                   Se viene eliminato il progetto (o modificato il cup), verrà eliminata (o aggiornata) anche la tupla in \textit{\takepart}
	                   \end{minipage}          \\[27pt]
	\hline
	lab\_code\_fk    & \begin{minipage}[t]{.6\textwidth}
		                   \raggedright
		                   Vincolo di chiave esterna.\\
		                   Se viene eliminato il laboratorio (o modificato il lab\_code), verrà eliminata (o aggiornata) anche la tupla in \textit{\takepart}
	                   \end{minipage} \\[36pt]
	\hline
\end{tabular}\bskip
\begin{tabular}{@{}| p{.3\textwidth} | p{.6\textwidth} |@{}} % works at
	\hline
	\multicolumn{2}{|c|}{\textbf{\worksat}}                                                                                                              \\
	\hline
	\textbf{Vincolo}  & \textbf{Descrizione}                                                                                                             \\
	\hline
	date\_integrity   & Verifica che la data in cui l'impiegato termina di lavorare al laboratorio, sia dopo la data di inizio.                          \\
	\hline
	cf\_base\_emp\_fk & \begin{minipage}[t]{.6\textwidth}
		                    \raggedright
		                    Vincolo di chiave esterna.\\
		                    Se viene eliminato l'impiegato (o modificato il cf), verrà eliminata (o aggiornata) anche la tupla in \textit{\worksat}
	                    \end{minipage}           \\[27pt]
	\hline
	lab\_code\_fk     & \begin{minipage}[t]{.6\textwidth}
		                    \raggedright
		                    Vincolo di chiave esterna.\\
		                    Se viene eliminato il laboratorio (o modificato il lab\_code), verrà eliminata (o aggiornata) anche la tupla in \textit{\worksat}
	                    \end{minipage} \\[36pt]
	\hline
\end{tabular}\bskip
\begin{tabular}{@{}| p{.3\textwidth} | p{.6\textwidth} |@{}} % works on
	\hline
	\multicolumn{2}{|c|}{\textbf{\workson}}                                                          \\
	\hline
	\textbf{Vincolo}        & \textbf{Descrizione}                                                   \\
	\hline
	check\_expiration\_date & Verifica che la data di fine contratto sia dopo la data di assunzione. \\
	\hline
	fk\_cf                  & Vincolo di chiave esterna.

	Se viene eliminato l'impiegato, verrà posta la chiave esterna a NULL.

	Questo perché si vuole salvare i contratti anche se un impiegato a progetto viene eliminato.     \\
	\hline
	fk\_cup                 & Vincolo di chiave esterna.                                             \\
	\hline
\end{tabular}\bskip
\begin{tabular}{@{}| p{.3\textwidth} | p{.6\textwidth} |@{}} % purchase
	\hline
	\multicolumn{2}{|c|}{\textbf{Domini}}                                 \\
	\hline
	\textbf{Vincolo} & \textbf{Descrizione}                               \\
	\hline
	emp\_type\_check & Il valore inserito dev'essere uno fra questi

	(' ', 'junior', 'middle', 'senior', 'manager')

	e non deve essere NULL                                                \\
	\hline
	check\_cup\_type & Il valore inserito dev'essere lungo $15$ caratteri \\
	\hline
	cf\_type\_length & Il valore inserito dev'essere lungo $16$ caratteri \\
	\hline
\end{tabular}

\newpage
\subsubsection{Vincoli inter-relazionali}\label{vincoli_inter_relazionali}
L'implementazione SQL dei seguenti trigger è riportata nella \hyperref[trigger]{\textcolor{red}{\underline{sezione dei trigger e trigger functions}}}\medskip
\begin{itemize}
	\item \textbf{check\_equipment\_funds}\\
	      Prima di acquistare un'attrezzatura, controlla che il progetto abbia i fondi.

	\item \textbf{check\_funds\_to\_hire}\\
	      Prima di assumere un impiegato a progetto, controlla che il progetto abbia i fondi.

	\item \textbf{check\_laboratory\_assignment}\\
	      Prima di inserire un laboratorio o prima di aggiornare il manager scientifico, controlla che questo sia \textit{senior}

	\item \textbf{check\_project\_assignment}\\
	      Prima di inserire un progetto, o prima di aggiornare il referente scientifico o il supervisore, controlla che questi siano rispettivamente \textit{senior} e \textit{manager}

	\item \textbf{check\_works\_on\_date}\\
	      Controlla che lo stesso impiegato non lavori già allo stesso progetto. Può lavorare allo stesso progetto solo se la data di assunzione del nuovo contratto è superiore alla data di scadenza del suo ultimo contratto per quel progetto.

	\item \textbf{crypt\_password}\\
	      Prima di inserire un \textit{impiegato}/\textit{impiegato a progetto}, cripta la password.

	\item \textbf{employee\_hiring}\\
	      Quando viene assunto un impiegato (ovvero quando viene inserito nella tabella \textit{\baseemp}), viene inserito un log in \textit{\careerlog}.\\
	      \null \quad (' ', 'junior')

	\item \textbf{fire\_emp}\\
	      Prima di licenziare un impiegato, ossia prima di inserire una tupla in \textit{\careerlog} dove il campo \textit{new\_role} = ' ' , controlla che l'impiegato non sia referente scientifico o manager di progetto, oppure che non sia manager scientifico di un laboratorio.

	      In quel caso non è possibile licenziarlo, bisognerà prima rimuoverlo da questi ruoli.

	\item \textbf{max\_lab\_in\_project}\\
	      Verifica che ad un progetto prendano parte al massimo $3$ laboratori

	\item \textbf{take\_part\_no\_duplicates}\\
	      Verifica che un laboratorio non prenda parte più volte allo stesso progetto nello stesso momento.

	\item \textbf{update\_emp\_type}\\
	      Quando viene inserita una tupla in \textit{\careerlog}, aggiorna il campo \textit{type} di \textit{\baseemp} in base al \textit{new\_role} appena inserito.

	\item \textbf{validate\_equipment\_request}\\
	      Prima di inserire una richiesta di attrezzatura, controlla che il laboratorio richiedente partecipi al progetto.

	\item \textbf{works\_at\_no\_duplicates}\\
	      Prima di inserire un impiegato in \textit{\worksat}, ossia prima di farlo lavorare in un laboratorio, controlla che lo stesso impiegato non lavori già allo stesso laboratorio.

	\item \textbf{emp\_works\_at\_one\_lab}\\
	      Controlla che un impiegato lavori ad un unico laboratorio


	\item \textbf{end\_project}\\
	      Quando un progetto termina, aggiorna l'\textit{end\_date} anche in \textit{\takepart}
\end{itemize}
